\chapter{打包小部件}
当创建一个应用程序到时候,你将会在窗口中放置多个小部件。我们第一个helloworld例子
仅仅使用了一个小部件,所以我们可以简单的使用gtk.Container.add()方法来‘pack’这个
小部件到窗口中。但是当想要在一个窗口中放置多于一个的小部件的时候,你又如何控制
小部件的安置呢?这就是packing的来源。
\section{Packing Boxes 的理论}
多数组装是通过创建盒。这些看不见的小工具容器,我们可以收拾我们的小部件到
有两种形式,一种横向盒和一个垂直的盒子。当包装部件到一个水平的盒子,对象是
水平插入由左到右或从右到左,具体取决于所使用的调用。在一个垂直的盒子,组装构件,
从顶部向底部或反之亦然。您可以使用任何其他框的里面或旁边的箱子组合创造
所需的效果。


要创建一个新的横向盒,我们可以通过调用gtk.HBox(),对于纵向盒,gtk.VBox().“
pack_start()和pack_end()方法用于将这些容器中的对象。 pack_start()
方法将开始在顶部,其工作方式,在一个vbox,并装在一个hbox左到右.pack_end()
方法将做相反的,包装在一个vbox从底部到顶部,然后从右到左横向盒。使用这些方法
允许右对齐,左对齐我们的小部件,并可以混合以任何方式,以达到预期的效果。我们将
在我们的例子中,使用pack_start()。的对象可以是另一个容器或构件.事实上,许多构件
本身就是容器,包括按钮,但我们通常只使用一个按钮,一个标签内.


通过使用这些调用,GTK知道在哪里,你要放置你的部件,所以它可以做自动调整大小和
其他漂亮的的东西。也有一些选择,如何包装你的构件。正如你可以想像,这种方法
给我们带来了相当多的灵活性时,放置和创建构件。
\section{Details of Boxes}
由于这种灵活性,在包装盒中的GTK第一印象可能是混乱的。有很多的选项,这是不
显而易见的,它们是如何组合在一起的。然而,归根结底,有五个不同的风格。图4.1
“包装:五变奏曲”,说明了结果,的运行的程序packbox.py的[例子/ packbox.py的]与
参数1:	

Figure 4.1. Packing: Five Variations

每一行包含一个水平盒子(盒子)与几个按钮。调用包是包调用的简写
每到横向盒的按钮。被打包到每个按钮的横向盒以同样的方式(即,相同的参数
pack_start()方法)。
这是pack_start()方法的一个例子。

box.pack_start(child, expand, fill, padding)

盒包装的对象中,您的第一个参数是孩子被包装物。对象将
所有的按钮了,所以我们会被包装成箱的按钮。
扩展参数pack_start()和pack_end()控制的部件是否在箱
填写包装盒中所有的额外空间,这样的框扩展以填充该区域配发予(真),或包装盒收缩
正好契合了小部件(假)。设置为False的扩展将允许你做你的左,右的理由
小部件。否则,他们将扩大到符合入禁区,可以达到同样的效果,通过只使用一个
pack_start()或pack_end()。

填充的包装方法控制参数是否额外的空间分配给对象本身(TRUE),
或微胖这些对象周围的框(假)。它只有一个效果,如果扩展参数也是真实的。
Python允许的方法或函数定义的默认参数值和参数关键字。始终
本教程中我将展示的功能定义和方法,违约和关键字以粗体显示(如适用)。为
例如pack_start()方法被定义为:

box.pack_start(child, expand=True, fill=True, padding=0)
box.pack_end(child, expand=True, fill=True, padding=0)
孩子,扩展,填充和填充的关键字。扩展,填充和填充参数
默认显示。子参数必须被指定。
当创建一个新的框,该函数看起来像这样:
hbox = gtk.HBox(homogeneous=False, spacing=0)
vbox = gtk.VBox(homogeneous=False, spacing=0)

参数homogeneous 到gtk.HBox()和控制包装盒中的每个对象是否有gtk.VBox()
相同的大小(即,相同的宽度在一个hbox,或以相同高度在一个vbox)。如果它被设置,包例程功能
基本上,如果展开争论始终是打开的。
间距(框创建时设置)和padding(设置元素的时候装)之间的区别是什么?
间距之间添加对象,和填充被加到任一侧上的一个对象。图4.2中,“包装与间距
和“填充”说明了差异;传递一个参数2 packbox.py [例子/ packbox.py的]:
\section{Packing Demonstration Program}
[examples/packbox.py]的代码,用于创建上面的图片。它的评论相当严重,所以我希望你
不会有任何问题。运行它自己和它一起玩。

开始一个简短的参观packbox.py [实例/ packbox.py]代码定义一个辅助函数用线14-50
make_box(),创建一个水平的盒子和填充按钮,根据指定的参数。一
参考水平中被返回。	


line 52-241定义PackBox1的类的初始化方法的__init__(),创建一个窗口和一个孩子的垂直
框,填充不同的部件安排取决于该参数传递给它。如果1被传递,
212-214线创建一个窗口,显示独特的包装时,可以使用不同的安排
均匀,扩展和填充参数。如果获得通过,2线140-182创建一个窗口,显示各种
组合充满间距和边距。最后,如​​果通过一个3线188-214创建一个视窗显示该
使用左对齐按钮和pack_end()方法来右对齐的标签的的pack_start()方法。线
215-235垂直的盒子包装成一个按钮,创建一个水平的盒子。按钮“点击”的信号是
连接到的PyGTK的main_quit()函数来终止程序。
行250-252检查命令行参数,并退出程序使用sys.exit()函数,如果没有
只有一个参数。线253创建一个PackBox1的的实例。 254行调用main()函数开始的GTK
事件处理循环。
在这个示例程序中,不会保存在该对象的实例引用的各种部件(除窗口)
,因为他们不需要的属性。

\section{Packing Using Tables}
让我们来看看包装的另一种方式 - 表。这些可以在某些情况下是非常有用的。
使用表格中,我们将创建一个网格,我们可以把英寸的部件可能需要尽可能多的空间,因为我们指定的窗口小部件。
看的第一件事,当然,是gtk.Table()函数:
table = gtk.Table(rows=1, columns=1, homogeneous=False)

均匀的说法有做表的盒子的大小。如果是真正的均匀,表
框的大小被调整到表中的最大的窗口小部件的大小。如果是假均匀,大小的表箱
取决于由最高的窗口小部件在相同的行中,最宽的窗口小部件在其列。
这些行和列的布局从0到n,其中n是在调用中指定的数字对gtk.Table()。所以,如果
你指定行= 2 = 2列,布局看起来像这样:

0	1	2
0+----------+----------+
|
|
|
1+----------+----------+
|
|
|
2+----------+----------+


请注意,在左上角开始的坐标系统。要放在一个盒子里放置一个小部件,请使用以下
方法:
table.attach(child, left_attach, right_attach, top_attach, bottom_attach, xoptions=EXPAND|FILL, yoptions=EXPAND|FILL, xpadding=0, ypadding=0)
的表实例是您创建的表与gtk.Table()。 (“子”)的第一个参数是你希望的部件
放置在表中。
left_attach,right_attach,top_attach和bottom_attach参数指定放置
小部件,有多少盒使用。如果你想有一个按钮在右下角我们的2x2表的表项,并希望它
填补了这一项目,left_attach = 1,right_attach = 2,top_attach = 1,bottom_attach
= 2。
现在,如果你想要一个小部件,占据整个我们的2x2表的最上面一行,你会使用left_attach = 0,
right_attach top_attach = 0,bottom_attach = 1 = 2,。
的xoptions和yoptions是用来指定包装的选项,并可能是按位进行逻辑或运算在一起以允许
多种选择。
这些选项包括:
FILL
SHRINK
EXPAND
If the table cell is larger than the widget, and FILL is specified, the widget will expand to
use all the room available in the cell.
If the table widget was allocated less space then was requested (usually by the user resizing
the window), then the widgets would normally just be pushed off the bottom of the window
and disappear. If SHRINK is specified, the widgets will shrink with the table.
This will cause the table cell to expand to use up any remaining space allocated to the table.


填充就像在盒子里,制定一个明确的以像素为单位指定窗口小部件周围地区。
我们也有set_row_spacing()和set_col_spacing()方法。这些附加的行之间的间距
指定的行或列。

table.set_row_spacing(row, spacing)
and
table.set_col_spacing(column, spacing)

请注意,为列,空间进入到右侧的列,和行,空间以下的行。
您还可以设置一致的所有行和/或列间距:
table.set_row_spacings(spacing)
and
talbe.set_col_spacings(spacing)

请注意,这些调用中,最后一排和最后一列没有得到任何间距。
\section{Table Packing Example}
示例程序table.py [的例子/ table.py]使一个窗口,一个2x2的表中有三个按钮。前两个
按钮将被放置在上排。第三,退出按钮,是摆在下面一行,跨越两列。
图4.4中,“包装”说明了出现的窗口中,使用表:	
Here's the source code:
表类中定义线9-78。第12-13行定义的回调()方法被调用时的两个
按钮“点击”。回调只是打印一条消息到控制台,表示哪个按钮被按下使用
传递的字符串数据。
第16-18行定义delete_event()方法被调用时,窗口定为删除
窗口管理器。
第20-78行定义表的实例初始化方法__init__()方法。它创建了一个窗口(22行),设置
窗口标题(第25行),连接delete_event()回调的“delete_event”信号(第29行),并设置
边框宽度(第32行)。甲gtk.Table在第35行中被创建,并添加到第38行中的窗口。
上部的两个按钮被创建(第41行和第55行),他们的“点击”的信号被连接到的回调()方法
(第45行和第59行),并连接到该表的第一行中(第49行和第61行)。第66-72行创建了“Quit”按钮,
“clicked”信号连接到的main_quit()函数,将其连接到跨越整个第二行的表。	

