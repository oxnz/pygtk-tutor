\chapter{Menu Widget}
pygtk中有两种方式生成菜单:有简单的也有困难的。都有自己的用户,但是你通常可以使用ItemFactory(简单方式)。‘hard’方式是直接使用调用生成所有菜单。简单的方式使用gtk.ItemFactory调用。这个一个很简单的,但是每个都有各自的优缺点。
Note
在PyGTK 2.4 ItemFactory 被废弃了-使用UIManager代替了。
ItemFactory使用起来更简单,并添加新的菜单,虽然写一些包装函数来
创建菜单使用手动的方法可以有更好的可用性。使用Itemfactory是不可以给菜单添加图像或'/'字符的。
\section{手工创建菜单} %Manual Menu Creation
在传统的实际教学中,我们首先向你展示复杂的方式.:)
有三个小工具,进入一个菜单栏和子菜单:
•一个菜单项,这是用户想要什么选择,例如,“保存”
•一个菜单,作为容器的菜单项,
•菜单栏,这是一个容器为每个单独的菜单。
这是稍微复杂的菜单项构件用于两个不同的东西。它们都是
部件组装到菜单,和被打包到菜单栏,选择时,激活的小工具
菜单。
让我们来看看,用于创建菜单和菜单栏的功能。这第一项功能是用来创建一个新的
菜单栏:
menu_bar = gtk.MenuBar()
这个相当自我解释的函数创建一个新的菜单栏。您可以使用gtk.Container的add()方法来收拾
这成窗口,或gtk.Box的包装方法,将它打包成箱 - 按钮一样。
menu = gtk.Menu()
这个函数返回一个对新菜单的引用;它从来都不会真正的显示(with the show() method),
它只是一个菜单项的容器。我希望你在看下面例子的时候能够理解的更清楚。

下一个函数是用于创建被打包进菜单(and menubar)的菜单项.
menu_item = gtk.MenuItem(label=None)
这个label,如果有的话,将被解析助记字符。这个调用被用于创建将要显示的菜单项。请记住,区分与gtk.Menu()和“菜单项的”菜单“创建”创建
由gtk.MenuItem()函数。该menu item将是一个真实的按钮执行相应的操作,而
menu将是菜单项的容器。

一旦你已经创建了一个菜单项,你必须把它放到一个菜单。这是通过使用append()方法。为了捕捉用户选择的菜单项,我们需要连接到“activate”信号通常的方式。所以,如果我们
要创建一个标准的File菜单的选项,具有打开,保存,和退出,代码看起来是这样的:
file_menu = gtk.Menu()	# Don't need to show menus

# Create the menu items
open_item = gtk.MenuItem("Open")
save_item = gtk.MenuItem("Save")
quit_item = gtk.MenuItem("Quit")

# Add them to the menu
file_menu.append(open_item)
file_menu.append(save_item)
file_menu.append(quit_item)

# Attach the callback functions to the activate signal
open_item.connect_object("activate", menuitem_response, "file.open")
save_item.connect_object("activate", menuitem_response, "file.save")

# We can attach the Quit menu item to our exit function
quit_item.connect_object("activate", destroy, "file.quit")

# We do need to show menu items
open_item.show()
save_item.show()
quit_item.show()
到现在我们就有了自己的菜单。现在我们需要创建一个menubar和一个menu item给File entry, 来添加我们的菜单。代码如下:
menu_bar = gtk.MenuBar()
window.add(menu_bar)
menu_bar.show()

file_item = gtk.MenuItem("File")
file_item.show()
现在我们需要把menu与file_item关联起来。这是通过如下方法:
menu_item.set_submenu(submenu)
所以,我们的例子接着是:
file_item.set_submenu(file_menu)
剩下的只是把menu添加到menubar,如下完成:
menu_bar.append(child)
在我们的例子中如下:
menu_bar.append(file_item)
如果我们想让菜单上的菜单,如帮助菜单往往是右对齐,我们可以使用下面的方法
(又是对file_item在当前的例子),然后将其连接到菜单栏。
menu_item.set_right_justified(right_justified)
这里总结了需要创建一个菜单栏,菜单,连接的步骤:
•创建一个新的菜单,使用gtk.Menu()
•使用多个gtk.MenuItem()的调用,你想对你的菜单中每个项目。使用的append()
的方法来把这些新项目的菜单。
•创建一个菜单项使用gtk.MenuItem()。这将是菜单的根,这里显示的文字会
在菜单栏上本身。
•使用set_submenu()方法来将菜单的根菜单项(在上面的步骤中创建的1)。
•使用gtk.MenuBar()创建一个新的菜单栏。此步骤只需要进行一次时创造了一系列的
一个菜单栏上的菜单。
•使用append()方法将根菜单上的菜单栏。
创建一个弹出菜单几乎是相同的。不同的是,菜单也不能由一个菜单栏,张贴“自动”,
但显式调用一个按钮按下的事件,例如在弹出()方法。采取以下步骤:
创建一个事件处理回调。它需要有格式:
def handler(widget, event):
•它会使用事件,找出弹出菜单。
•在事件处理程序中,如果该事件是一个鼠标键按下,处理事件作为一个按钮事件(它是),并把它作为
所示的示例代码中,将信息传递给弹出()方法。
•该事件处理程序绑定到一个小部件:
widget.connect_object(“事件”的处理程序,菜单)
其中widget是要绑定到的部件,处理器的处理功能,菜单是菜单创建
gtk.Menu()。这可能是一个菜单,其中还贴了一个菜单栏,如图所示的示例代码中。
\section{手工创建菜单的例子} %Manual Menu Example
有关做到这一点。让我们一起来看看在的menu.py [的例子/ menu.py]例如方案,以帮助澄清
的概念。图11.1中,“菜单示例”说明了程序显示:
The menu.py [examples/menu.py] program source code is:

\section{使用ItemFactory}
现在,我们已经向你展示了hard way,这里将介绍如何使用gtk.ItemFactory的调用。
\section{使用ItemFactory的例子}
[例子/ itemfactory.py的,itemfactory.py]示例程序使用的gtk.ItemFactory的,。图11.2“项目
厂范例“说明了程序显示:
The source code for itemfactory.py [examples/itemfactory.py] is:
For now, there’s only this example. An explanation and lots ’o’ comments will follow later.

