\documentclass[a4paper,11pt]{article}
\usepackage{xspace}
\newcommand{\TUG}{\textsc{Tex Users Group}\xspace}
\newcommand{\keyword}[1]{\textbf{#1}}
%\usepackage[colorlinks=true,linkcolor=red]{hyperref}
\begin{document}
	\title{Example 3}
	\author{My name}
	\date{January 5, 2012}
	\maketitle
	\section{What's this?}
	This	is    our
	second document.

	It contains two paragraphs. The first line of a paragraph will be
	indented, but not when it follows a heading.
	% Here's a comment
	
	Text can be \emph{emphasized}

	Besides begin \textit{italic} words could be \textbf{bold},
	\textsl{slanted} or typeset in \textsc{Small Caps}.

	Such commands can be \textit{\textbf{nested}}.

	\emph{See how \emph{emphasizing} looks when nested.}
	\section{\textsf{\LaTeX\ resources on the internet}}
	The best place for downloading LaTeX related software is CTAN.
	Its address is \texttt{http://www.ctan.org}.
	\section{\sffamily\LaTeX\ resources in the internet}
	The best place for downloading LaTeX related software is CTAN.
	Its address is \ttfamily http://www.ctan.org\rmfamily.
	\section{text size}
	\noindent\tiny We \scriptsize start \footnotesize \small small,
	\normalsize get \large big \Large and \LARGE bigger, \huge huge
	and \Huge gigantic!
	\section{The \TUG}
	The \TUG is an organization for people who are interested in \TeX
	or \LaTeX.
	\section{more universal command}
	\keyword{Grouping} by curly braces limits the \keyword{scope} of
	\keyword{declarations}.
	\section{box}
	\parbox{3cm}{\footnote{TUG is an acronym.} It means \TeX\ Users Group.}
\end{document}
