\chapter{Scribble, A Simple Example Drawing Program}
\section{Scribble Overview}
在这一节中,我们将创建一个简单的绘图程序。在这个过程中,我们将检测如何处理鼠标事件,如何绘制一个窗口和如何通过使用a backing pixmap使得绘图更好。
\section{Event Handling}
我们已经讨论过的GTK +信号是高层次的行动,如一个菜单项被选中。但是,
有时它是有用的了解较低级别的事件,例如鼠标移动,或一个键被按下。
也有GTK +信号对应到这些低级别的事件。这些信号的处理程序有一个额外的
参数这是一个gtk.gdk.Event的对象,其中包含有关事件的信息。例如,运动事件
处理程序被通过一个gtk.gdk.Event的的对象的含EventMotion信息(部分)属性
如:
type
window
time
x
y
...
state
...
window是窗口,在其中发生的事件。
x和y给出的事件的坐标。
type将被设置到的事件类型,在这种情况MOTION_NOTIFY。类型(在模块gtk.gdk)是:
NOTHING
DELETE
DESTROY
EXPOSE
a special code to indicate a null event.
the window manager has requested that the toplevel window be
hidden or destroyed, usually when the user clicks on a special
icon in the title bar.
the window has been destroyed.
all or part of the window has become visible and needs to be
redrawn.
MOTION_NOTIFY the pointer (usually a mouse) has moved.
BUTTON_PRESS a mouse button has been pressed.
_2BUTTON_PRESS a mouse button has been double-clicked (clicked twice within
              a short period of time). Note that each click also generates a
	                   BUTTON_PRESS event.
			   _3BUTTON_PRESS a mouse button has been clicked 3 times in a short period of
			                 time. Note that each click also generates a BUTTON_PRESS event.
					 BUTTON_RELEASE
					 KEY_PRESS
					 a mouse button has been released.
					 a key has been pressed.
					 KEY_RELEASE a key has been released.
					 ENTER_NOTIFY the pointer has entered the window.
					 LEAVE_NOTIFY the pointer has left the window.
					 FOCUS_CHANGE the keyboard focus has entered or left the window.
					 CONFIGURE
					 changed.
					 windows.
					 MAP
					 UNMAP
					 the size, position or stacking order of the window has
					 ←
					 Note that GTK+ discards these events for GDK_WINDOW_CHILD
					 ←
					 the window has been mapped.
					 the window has been unmapped.
PROPERTY_NOTIFY a property on the window has been changed or deleted.
SELECTION_CLEAR the application has lost ownership of a selection.
SELECTION_REQUEST another application has requested a selection.
SELECTION_NOTIFY a selection has been received.
PROXIMITY_IN an input device has moved into contact with a sensing surface
            (e.g. a touchscreen or graphics tablet).
	    PROXIMITY_OUT an input device has moved out of contact with a sensing
	    surface. ←
	    DRAG_ENTER the mouse has entered the window while a drag is in progress.
	    DRAG_LEAVE the mouse has left the window while a drag is in progress.
	    DRAG_MOTION the mouse has moved in the window while a drag is in progress.
	    DRAG_STATUS the status of the drag operation initiated by the window has
	    changed. ←
	    DROP_START a drop operation onto the window has started.
	    DROP_FINISHED the drop operation initiated by the window has completed.
	    CLIENT_EVENT a message has been received from another application.
	    VISIBILITY_NOTIFY
	    NO_EXPOSE
	    when parts
	    the window visibility status has changed.
	    indicates that the source region was completely available
	    ←
	    of a drawable were copied. This is not very useful.
	    SCROLL
	    ?
	    WINDOW_STATE
	    SETTING
	    ?
	    ?
	    事件发生时的状态指定在修改状态(即,它指定哪一个修改键和鼠标
	    按钮被按下)。它是按位或以下的一些(模块gtk.gdk):
	    SHIFT_MASK
	    LOCK_MASK
	    CONTROL_MASK
	    MOD1_MASK
	    MOD2_MASK
	    MOD3_MASK
	    MOD4_MASK
	    MOD5_MASK
	    BUTTON1_MASK
	    BUTTON2_MASK
	    BUTTON3_MASK
	    BUTTON4_MASK
	    BUTTON5_MASK
	    至于其他信号,以确定当事件发生时,我们调用connect()方法时,会发生什么情况。但是,我们也
	    需要让GTK +的事件,我们希望得到通知的。要做到这一点,我们所说的方法:	
widget.set_events(events)
事件参数指定我们感兴趣的事件,它是按位或指定不同的常量
类型的事件。对于未来的参考,的事件类型(模块gtk.gdk)的有:
EXPOSURE_MASK
POINTER_MOTION_MASK
POINTER_MOTION_HINT_MASK
BUTTON_MOTION_MASK
BUTTON1_MOTION_MASK
BUTTON2_MOTION_MASK
BUTTON3_MOTION_MASK
BUTTON_PRESS_MASK
BUTTON_RELEASE_MASK
KEY_PRESS_MASK
KEY_RELEASE_MASK
ENTER_NOTIFY_MASK
LEAVE_NOTIFY_MASK
FOCUS_CHANGE_MASK
STRUCTURE_MASK
PROPERTY_CHANGE_MASK
VISIBILITY_NOTIFY_MASK
PROXIMITY_IN_MASK
PROXIMITY_OUT_MASK
SUBSTRUCTURE_MASK
有一些细微的问题时要注意调用方法set_events()来。首先,它必须被称为
之前的X窗口的PyGTK的部件的创建。在实际应用中,这意味着你应该马上把它
在创建小部件。其次,小部件必须是1,这将实现与相关联的X窗口中。为
效率,许多类型的组件没有自己的窗口,但吸引他们的父窗口。这些部件包括:
gtk.Alignment
gtk.Arrow
gtk.Bin
gtk.Box
gtk.Image
gtk.Item
gtk.Label
gtk.Layout
gtk.Pixmap
gtk.ScrolledWindow
gtk.Separator
gtk.Table
gtk.AspectFrame
gtk.Frame
gtk.VBox
gtk.HBox
gtk.VSeparator
gtk.HSeparator
为了捕捉这些小部件的事件,你需要使用一个事件盒构件。请参见第10.1节“事件盒的”widget
了解详细信息。
所设定的PyGTK的每种类型的事件的事件属性是:
every event
type
window
send_event
NOTHING
DELETE
DESTROY
# no additional attributes
EXPOSE area
      count
      MOTION_NOTIFY time
                   x
		               y
			                  pressure
					            xtilt
						             ytilt
							             state
								            is_hint
									          source
										       deviceid
										           x_root
											      y_root
											      BUTTON_PRESS
											      _2BUTTON_PRESS
											      _3BUTTON_PRESS
											      BUTTON_RELEASE
											      time
											      x
											      y
											      pressure
											      xtilt
											      ytilt
											      state

										button
										source
										deviceid
										x_root
										y_root
										KEY_PRESS
										KEY_RELEASE
										ENTER_NOTIFY
										LEAVE_NOTIFY
										FOCUS_CHANGE
										CONFIGURE
										time
										state
										keyval
										string
										subwindow
										time
										x
										y
										x_root
										y_root
										mode
										detail
										focus
										state
										_in
										x
										y
										width
										height
										MAP
										UNMAP
										PROPERTY_NOTIFY
										# no additional attributes
										atom
										time
										state
										SELECTION_CLEAR
										SELECTION_REQUEST
										SELECTION_NOTIFY
										selection
										target
										property
										requestor
										time
										PROXIMITY_IN
										PROXIMITY_OUT
										time
										source
										deviceid
										DRAG_ENTER
										DRAG_LEAVE
										DRAG_MOTION
DRAG_STATUS
DROP_START
DROP_FINISHED
CLIENT_EVENT
VISIBILTY_NOTIFY
context
time
x_root
y_root
message_type
data_format
data
state
NO_EXPOSE
# no additional attributes

	\subsection{Scribble-Event Handling}
我们的绘图程序,当按下鼠标按钮,当鼠标移动时,我们想知道的,所以我们
指定POINTER_MOTION_MASK和BUTTON_PRESS_MASK。我们也想知道,当我们需要重绘
我们的窗口,所以我们指定EXPOSURE_MASK。尽管我们想通过配置事件通知时,我们
窗口大小的变化,我们不指定相应STRUCTURE_MASK标志,因为它是自动
指定的所有窗口。
事实证明,但是,有一个问题是,与只指定POINTER_MOTION_MASK。这将导致
服务器添加一个新的运动事件到事件队列每次用户移动鼠标。试想一下,它需要我们0.1
秒处理一个运动事件,但是X服务器队列中一种新的运动事件,每0.05秒。我们将很快得到
用户绘图的方式落后。如果用户利用5秒钟,它会带我们再过5秒后,他们
释放鼠标按钮!我们想的是只得到一个运动事件,每个事件我们处理。的方式来
这是到指定POINTER_MOTION_HINT_MASK。
当我们指定POINTER_MOTION_HINT_MASK,服务器发送一个运动事件第一时间的指针
移动后,进入我们的窗口,或在一个按钮按下或释放事件。随后的运动事件将被抑制
除非我们明确要求位置的指针使用gtk.gdk.Window方法:
x,y,mask = window.get_pointer()
窗口是一个gtk.gdk.Window对象。 x和y的坐标的指针和掩模是修饰符掩码
检测哪个键被按下。 (有一个gtk.Widget的方法,get_pointer的(),这提供了相同的
信息作为gtk.gdk.Window get_pointer的()的方法,但它不返回的掩模信息)
该scribblesimple.py [的例子/ sc​​ribblesimple.py]示例程序演示了基本的使用事件和事件
处理程序。图24.2,“简单的涂鸦示例”说明了在行动方案:
The event handlers are connected to the drawing_area by the following lines:
# Signals used to handle backing pixmap
drawing_area.connect("expose_event", expose_event)
drawing_area.connect("configure_event", configure_event)
# Event signals
drawing_area.connect("motion_notify_event", motion_notify_event)
drawing_area.connect("button_press_event", button_press_event)
drawing_area.set_events(gtk.gdk.EXPOSURE_MASK
| gtk.gdk.LEAVE_NOTIFY_MASK
| gtk.gdk.BUTTON_PRESS_MASK
| gtk.gdk.POINTER_MOTION_MASK
| gtk.gdk.POINTER_MOTION_HINT_MASK)
的button_press_event()和motion_notify_event()事件处理程序中scribblesimple.py [考试
	普莱斯/ scribblesimple.py是:
def button_press_event(widget, event):
if event.button == 1 and pixmap != None:
draw_brush(widget, event.x, event.y)
return True
def motion_notify_event(widget, event):
if event.is_hint:
x, y, state = event.window.get_pointer()
else:
x = event.x
y = event.y
state = event.state
if state & gtk.gdk.BUTTON1_MASK and pixmap != None:
draw_brush(widget, x, y)
return True
The expose_event() and configure_event() handlers will be described later.

\section{The DrawingArea Widget, And Drawing}
