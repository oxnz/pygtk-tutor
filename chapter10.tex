\chapter{Container Widgets}
\section{The EventBox}
一些GTK构件并没有相关联的X窗口,所以他们只是利用他们的父母。正因为如此,他们不能
接收事件,如果大小是不正确的,他们不夹,这样你就可以得到凌乱的覆盖等,如果你需要
从这些小部件,事件盒,是给你的。
乍一看,事件盒构件可能出现的是完全无用的。它在屏幕上绘制任何东西和响应
没有任何事件。然而,它的功能 - 它提供了一个X窗口,其子部件。这是很重要的
尽可能多的GTK构件没有相关联的X窗口。没有一个X窗口,可以节省内存和提高
的性能,但也有一些缺点。一个小部件没有X窗口不能接收事件,不执行
其内容和任何剪裁可以不设置它的背景颜色。虽然名称事件盒强调
事件处理功能,窗口小部件也可以被用于裁剪。 (更多信息,请参阅下面的示例)。
要创建一个新的事件盒构件,使用:
event_box = gtk.EventBox()
A child widget can then be added to this event_box:
event_box.add(widget)
[的例子/ eventbox.py]的eventbox.py的示例程序演示了这两种用途的事件盒 - 一个标签
创建剪切到一个小盒子,有一个绿色的背景和设置,以便在标签上点击鼠标,使
程序退出。调整窗口的大小显示不同数量的标签。图10.1“事件”框中例“
说明了程序显示:	
The source code for eventbox.py [examples/eventbox.py] is:
\section{The Alignment widget}
\section{Fixed Container}
\section{Layout Container}
\section{Frames}
\section{Aspect Frames}
\section{Paned Window Widgets}
\section{Viewports}
\section{Scrolled Windows}
\section{Button Boxes}
\section{Toolbar}
\section{Notebooks}
\section{Plugs and Sockets}
\subsection{Plugs}
\subsection{Sockets}
